\chapter{Introduction} \label{chap:chap1}

% Contextualization (goal: 1/2 page)

The XIX century marked a significant shift in the world's perception of energy resources as the desire to invest in renewable energy sources to power modern societies grew. This transition was driven by the need to reduce dependency on fossil fuels, mitigate the effects of global warming, and slow climate change. Renewable energy sources offer a range of benefits, including reduced greenhouse gas emissions, increased energy security, and air quality. Solar photovoltaic energy is a desirable renewable energy source due to its abundance, accessibility, and environmental benefits. While solar photovoltaic energy has proven to be both cost-efficient and environmentally friendly, it also comes with unprecedented challenges, such as its intermittent nature, low electrical inertia, complex forecasting, and geographic-dependent operating conditions. Despite these challenges, recent reports \cite{cap} show that the economic benefits of investing in renewable energy outweigh the complications, as there is an increasing global investment trend in these sources.

% Motivation (goal: 2 paragraphs)
The general construction of PV farms, particularly on the utility-scale, has led to a need for effective maintenance and monitoring to ensure maximum efficiency and operational reliability. Towards this, various algorithms and routines are used to monitor the state of PV farms and identify any potential issues that may arise. Fault detection is crucial to this process, allowing PV farm operators to identify and address problems quickly. Detecting faults and identifying the necessary steps can prevent or minimize downtime and ensure optimal performance. Given the importance of maintaining high levels of operation, knowing if action is needed to restore or fix components from an anomalous scenario is desirable for reducing investment risk and maximizing profits.

Integrating intermittent energy resources into modern electric grids has led to stricter requirements for connecting such power systems to ensure safe grid operating conditions. As a result, companies that own or plan to build photovoltaic farms must comply with these requirements and have adequate power electronics and monitoring/control capabilities. Failure to meet these requirements can result in sanctions or fines for the responsible party, as well as potential impacts on system availability, asset value, and disturbance propagation to the grid. To minimize these risks and maximize the value of their assets, companies may opt to implement fault detection and state estimation tools. These tools allow for the early detection and resolution of potential issues and can prevent or minimize downtime. The need to create or improve existing fault detection and state estimation tools, and the search for the most effective methodologies for addressing these issues, drive research in this field.

% Investigation questions
Having laid the basis for why there must be system behavior assessment in utility-scale PV plants, it is necessary to understand what business concepts are crucial to this field. In the course of this work, the presented topics will go over the following questions:

\begin{itemize}
    \item What components mostly fail in photovoltaic power systems?
    \item What is the average frequency of faults?
    \item What fault detection/state estimation tools exist for photovoltaic
    power systems?
    \item What are the most successful ones?
    \item What's their structure? Are they mostly centralized or decentralized?
    \item What are their computational costs/efficiency?
    \item What is the expected magnitude of precision and confidence?
    \item Which key performance indicators can evaluate the success of these
    tools? 
    \item What are their implementation difficulties?
\end{itemize}

% Goals
With these questions uncovered, the main objective is to adapt or design a novel algorithm/approach to fault detection based on modern artificial intelligence solutions. However, this can be split into finer goals:

\begin{itemize}
    \item  Identify and study existing fault detection tools for photovoltaic
    power systems.
    \item  Adapt or develop a new tool.
    \item Apply and test the new tool in real case study PV assets.
    \item Validate the developed methodologies by comparison to reference tools.
\end{itemize}

% Methodology
Before reviewing state-of-the-art fault detection tools, types of failures in photovoltaic systems need to be understood: find which components usually fail, which ones fail more often, and how often. For this, it is necessary to understand such components' physical and electrical properties and the modeling techniques used to characterize them. There will be an assessment of utility-scale power plants architecture through literature, alongside the detection objective of state-of-the-art fault detection tools applied in this field. Then, there shall be an extensive analysis and review of what tools have been designed and used in this field. In this step, critical evaluation of the literature is a must for understanding the tool's scope, ease of implementation, and understanding that the data sets available for this work are compatible. Having selected the most prominent ones, they're to be qualitatively and quantitatively compared to each other in their application context so that the results allow objective evaluations. This process requires implementing these tools, following the guidelines in the respective article/book/report, verifying their metrics, and checking if the achieved results resemble the same as the literature suggests. It will require gathering data sets, which can either be artificially generated through simulation or provided by an enterprise that services photovoltaic plant owners.


% Impact and expected results
There's a desire that, in the end, the developed work helps achieve an improved method for fault detection and state estimation in photovoltaic power systems, resulting in a production-ready software application agile enough to deploy for multiple PV assets. It's intended that the algorithm specializes in data cohesion as a means of anomaly inference, allowing asynchronous and self-healing data transfers between the considered components. Depending on the new algorithm's characteristics, it could result in an approach capable of generalization and application to other engineering systems, benefiting more than just PV systems. No matter the chosen methodology, fault detection will, in most cases, result in an economic benefit, catastrophe prevention, and safety increase.