\chapter{Introdução} \label{chap:intro}

Este documento ilustra o formato a usar em dissertações na \Feup.
São dados exemplos de margens, cabeçalhos, títulos, paginação, estilos
de índices, etc. 
São ainda dados exemplos de formatação de citações, figuras e tabelas,
equações, referências cruzadas, lista de referências e índices.
Este documento não pretende exemplificar conteúdos a usar.

Uma recolha sobre as normas existentes pode ser encontrada em~\citet{kn:Mat93}.

Neste primeiro capítulo ilustra-se a utilização de citações e de
referências bibliográficas.

\section{Context} \label{sec:context}

The XIX century represents a substantial change to how the world perceives
energy resources, characterized by the urge to invest in renewable energy to 
power modern societies. These relatively modern alternatives are
proving to be both cost-efficient and environmentally friendly...

\section{Motivation} \label{sec:motivation}

Given the widespread increase in the construction of PV (Photovoltaic) farms, there
comes a need to maintain these facilities operational and yielding at maximum
efficiency. For that to be achieved, a series of algorithms must be executed
routinely to maintain the state estimation of a farm, as it is
desirable to know if any action must be taken to restore or fix components from
an anomalous scenario. Detecting and predicting component faults allows the
responsible agents to perform reparative or preventive maintenance on the
equipment as soon as possible, which in turn minimizes any potential losses by
increasing system availability.

With the continuous effort of integrating intermittent energy resources into
modern electric grids, the requirements for connecting such power systems have
been increasingly stricter to maintain safe grid operating conditions.
Consequently, companies that either possess or plan to build photovoltaic farms
must follow these requirements, which not only come with the need for adequate
power electronics but also monitoring/control capability.
\textcolor{red}{(INSERT SOURCES!)}
Since system faults can incur unpredicted electrical production behavior,
sanctions or fines might be applied to the responsible party. Keeping that in
mind, the previously mentioned preventive maintenance might be critical to also
minimize potential disturbance propagation to the grid, and avoid the
corresponding fines.

\section{Goals} \label{sec:goals}

The main scope of this paper is to apply deep learning techniques to diagnose
the operation of PV farms. It is desired to achieve the following:

\begin{itemize}
  \item Identify and study the existing fault prediction/detection tools
  \item Adapt and/or develop a new tool based on distributed architecture AI
  \item Apply and test the new tool on a real case study
  \item Validate methodologies through benchmarking and reference tools
\end{itemize}


\section{Estrutura da Dissertação} \label{sec:struct}

Para além da introdução, esta dissertação contém mais x capítulos.
No capítulo~\ref{chap:sota}, é descrito o estado da arte e são
apresentados trabalhos relacionados. 
No capítulo~\ref{chap:chap3}, ipsum dolor sit amet, consectetuer
adipiscing elit.
No capítulo~\ref{chap:chap4} praesent sit amet sem. 
No capítulo~\ref{chap:concl}  posuere, ante non tristique
consectetuer, dui elit scelerisque augue, eu vehicula nibh nisi ac
est. 
