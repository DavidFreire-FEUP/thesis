\chapter{Preliminary Work Plan} \label{chap:chap3}

Será recolher dados, implementar os modelos mais interessantes e relevantes encontrados na literatura e estudar as aplicações de cell neural networks para adaptar a deteção de falhas/coesão de dados. Capítulo por desenvolver...

% TODO

\chapter{Conclusions}

% TODO

With the contemporary state of investment in grid-connected big-scale photovoltaic systems, we can affirm that resource allocation to fault recovery is essential both to meet grid code requirements, safety and health standards, reduce investment risk and maximize the asset's throughput. The literature confirms such claim, by presenting a plenitude of research work regarding this problematic. Throughout its review, it is noted that there is a sparse variety of different methodologies all applied for both fault detection and classification on PV systems, from image-based to electrical-based, using signal processing, graph theory, statistics, machine learning, deep learning and even quantum machine learning. Different authors utilize the same dataset and research facility in various fault detection research works, which has the benefit of data consistency across multiple pieces of literature. However, there is a significant downside, as this work's scope encompasses PV systems with differing characteristics: utility-scale instead of small-scale, with fewer monitoring capabilities.