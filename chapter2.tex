\chapter{Fault detection in Utility Scale Photovoltaic Plants} \label{chap:chap2}


\section{Utility-Scale Photovoltaic System's Architecture}

Utility-scale photovoltaic (PV) power plants are large-scale systems connected to the electrical grid, having installed capacities ranging from kilowatts peak (kWp) to megawatts peak (MWp). These systems typically consist of many PV panels interconnected through power electronics to aggregate and inject power into the grid. The number and type of components in a PV power plant depend on the plant's scale and topology, with different configurations possible for large-scale applications, including central inverters, string inverters, and multi-string inverters \cite{lspv}. The physical installation of PV modules can include solar tracking apparatuses, such as single and dual-axis trackers \cite{Mourad2022}, which add to system complexity and change production behavior. Understanding the architecture and components of PV power plants is vital for designing, operating, and maintaining these systems, as it helps optimize their performance and reliability.

\begin{figure}[h!]
    \centering
    \includegraphics[width=15cm]{figures/chapter2/pvplant.drawio.pdf}
    \caption{Representation of utility-scale PV plant components and some possible faults.}
    \label{fig:topologies}
\end{figure}

Figure \ref{fig:topologies} presents a typical utility-scale PV plant architecture using the central inverter (or possibly multi-string inverter) configuration. It is noticeable that many system components may fail in one or more ways, which is why monitoring and fault detection algorithms are essential to maintain state estimation. The main subsystems considered in this work are the following:

\begin{itemize}
    \item Solar photovoltaic panels (with or without bypass diodes).
    \item Tracking mount.
    \item Electrical cabling.
    \item Inverter(s) (mostly with Max Power Point Trackers).
    \item AC Transformer(s).
    \item Protection components (circuit breakers, fuses, surge protectors,
    etc.)
\end{itemize}

Most of these components have intrinsic variables, such as voltage and current values, that can help determine their operation states. Given that the utility grids (and the associated electricity market) integrate large-scale PV assets, some of the before-mentioned components require constant monitoring and control, achieved with adequate embedded systems and sensor infrastructure \cite{AIPV}. Since monitoring utility-scale PV assets relies on the investment and technologies employed, engineers must consider data availability when developing data-driven algorithms. Thanks to the continuous advancements in communication technologies, namely in IoT (Internet Of Things), data acquisition is becoming faster, more reliable, and more precise. Not only is this fundamental for real-time asset assessment, but it also allows better training of fault detection algorithms. However, on the industrial scale (in the order of MWp production), having sensors embedded in every PV module comes with a high economic cost. Inverters are the components that usually possess monitoring capabilities, though the grid-tie connection should also be equipped with sensors. These can be considered the primary sources of information from utility-scale PV plants, with the most accurate, fast, and reliable data acquisition.

\begin{figure}[h!]
    \centering
    \includegraphics[width=\linewidth]{figures/chapter2/pvdata.drawio.pdf}
    \caption{Typical data flow of utility-scale PV power plants.}
    \label{fig:pvdataflow}
\end{figure}

Figure \ref{fig:pvdataflow} represents a simplified data flow representation of a grid-tied PV system's most commonly available state variables, with most of them suggested by the IEC 61724 standard \cite{iec61724}. An external meteorological data source is defined since the PV system manager usually needs climate information for (at least) forecasting purposes.

% TODO talk about typical data acquisition frequency?

\section{Faults in Photovoltaic Systems}

Several types of faults can occur in utility-scale photovoltaic (PV) power plants, which impact the performance and reliability of the system negatively. Unfortunately, some are very challenging to detect and protect the electrical installation against, thus requiring sophisticated detection algorithms \cite{Pillai2018}. Besides the economical price, their occurrence may even cause safety hazards, such as fires \cite{Alam2015}, thus the urgency in early detecting or preventing such events.

According to \cite{Pillai2018}, these faults can fit into three categories: electrical, mechanical, and environmental. Electrical faults include short circuits, open circuits, and inverter failure, affecting the PV panels' power output and the system's overall efficiency. Mechanical faults include broken panels, damaged cables, and defective inverters, which can lead to system downtime and reduced performance (although not mentioned, solar tracker failures could also belong in this category). Environmental faults include extreme weather events, such as hail or strong winds, which can damage the PV panels and other components \cite{faults}.

The authors in \cite{Hong2022} cover a comprehensive review of most types of faults studied in the ambit of detection and classification algorithms. However, authors in \cite{Livera2019} have a more succinct fault categorization that better fits this work's scope. They categorize all the major PV system faults into either DC-side or AC-side. Figure \ref{fig:faults} represents this detailed categorization with a tree-like structure.

Although also prone to failure, most literature on fault detection and classification for photovoltaic systems does not encompass solar tracking faults: most studies cover fixed PV systems. The supervision and assessment of these subsystems' correct functioning can be sensor-based \cite{Stepanov2014} or image-based. Some authors developed fault detection methods for these apparatuses \cite{Amaral2021}, using image processing on aerial photography to determine modules' slopes. This category of failures should be better supported when developing electrical data-driven algorithms since they can significantly affect the system's efficiency. Hence, this work will attempt to include said fault category in the proposed fault detection methodology.

\begin{figure}[h!]
    \centering
    \includegraphics[width=14cm]{figures/chapter2/faults.png} \caption{"Failures in grid-connected PV systems."} Image source and copyright: \cite{Pillai2018}.
    \label{fig:faults}
\end{figure}

Throughout the literature \cite{Braun2011}, some of the most noted faults in the context of fault detection are:

\begin{itemize}
    \item Shading: partial coverage, temporary or not, of a PV array or module. It might result in a Hot Spot fault.
    \item Soiling: dirt accumulation, blocking sunlight from reaching PV Cells. It might also result in a Hot Spot fault.
    \item Short circuit: either line-line or line-ground.
    \item Open circuit: connection breakage between modules.
    \item DC arc fault: electricity plasma arc formed on broken connections.
\end{itemize}

According to a 2017 survey conducted on five utility-scale PV plants in Italy \cite{Grimaccia2017}, the authors observed failure rates from <1\% to 3\% in the majority of plants and 81.8\% in the worst scenario. The high failure rate of the latter had a demonstrated cause that originated from manufacturing mistakes: snail trails. Besides this phenomenon, hot spot faults and bypass diode faults/disconnections were among the most common.

Alongside manufacturing failures, installation, planning, and other external effects can be the root cause for many of the presented faults \cite{sunny}.

\begin{figure}[h]
    \centering
    \includegraphics[width=8cm]{figures/chapter2/chartfailsurvey.png} \caption{"Circle chart related to the module defects in the 5 plants (over the total number of failures)."} Image source and copyright: \cite{Grimaccia2017}.
    \label{fig:faultchart}
\end{figure}

Having the distribution of fault types from real-life scenarios is quite helpful for formulating fault detection algorithms. It allows for better generation/selection of training data and class decisions. In figure \ref{fig:faultchart}, it is possible to observe the failure type distribution for 24.254 inspected modules. Soiling, shading, and mechanically related failures were not as prominent, with only a group share of around 6\%. It is relevant to note that discoloration represents almost a quarter of all faults.

Although the study had a limited geographic scope, with only a few power plants diagnosed, it allows for a more realistic view of the common scenarios encountered in typical operational ground-mounted utility-scale PV power plants.

Due to the difficulty of classifying some of these faults, given their similarity on the consequent effect in the system, it will be seen in further sections that most fault detection algorithms only endeavor to classify between two to five types of reviewed faults.

\section{Modeling photovoltaic's physical/electrical behavior}

Photovoltaic cells are the fundamental components of photovoltaic panels. They are made from semiconductor materials, such as silicon,  and absorb photons that generate an electric current. Their electrical behavior is characterizable using the current-voltage (I-V) equation \ref{eq:iv}. This equation, which represents a fundamental relationship governing the operation of PV cells, can be used to predict their performance under various operating conditions, such as differing solar irradiance and temperatures.

\begin{equation} \label{eq:iv}
    I = I_{ph} - I_d \times (e^{\frac{q \times (V_{pv} + I_{pv} \times R_{s})}{n \times k \times T}} - 1) - \frac{V_{pv} + I_{pv} \times R_s}{R_p}
\end{equation}

$I_{ph}$ (A) is the light-generated current;
$I_{0}$ (A) is the reverse saturation current;
$V_{pv}$ is the module's terminal voltage;
$I_{pv}$ is the module's output current;
$R_{s}$ ($\Omega$) is the series resistance;
$R_{p}$ ($\Omega$) is the shunt resistance;
$n$ (adimensional) is the diode ideality factor;
$k$ (J/K) is the Boltzman constant;
$T$ (K) is the cell temperature;
$q$ (C) is the electron charge;

For state estimation, it is crucial to accurately model PV modules' performance from the DC side of power converters. This information is vital for designing and optimizing PV power systems, as it enables predicting PV module performance under different conditions, as mentioned before. Accurate PV module models are also essential for state estimation and fault detection, as they provide critical information about the health and performance of PV modules, allowing for early identification of potential issues. In addition, they can be used to optimize the control and operation of PV power systems, which can improve the overall efficiency and reliability of the system \cite{Braun2011}.

Physical and empirical models broadly categorize the several state-of-the-art methods for modeling photovoltaic modules \cite{Braun2011}. Physical models lie on the fundamental physical principles governing PV modules' operation. They typically require detailed knowledge of the PV module's electrical and optical properties, such as its current-voltage (I-V) characteristics, spectral response, and temperature dependence. These models can accurately predict the PV module's performance under a wide range of operating conditions, but they may be complex and computationally intensive to implement \cite{Kumar2019}. On the other hand, empirical models are based on experimental data and are typically more straightforward to implement. However, they may not be as accurate as physical models, especially under conditions that differ significantly from those used to generate the experimental data (usually STC) \cite{Braun2011}. Some examples of state-of-the-art physical models for PV modules include the single-diode model (also known as the five-parameter model), and the two-diode model \cite{Godina2017}. In contrast, one of the most used state-of-the-art empirical models is the Sandia model \cite{Braun2011}. The choice of modeling method will depend on the specific application and the required level of accuracy and complexity; in some cases, there can be a combination of physical and empirical models.

Suppose the need arises to model PV modules in this work. In that case, it is critical to select a simple methodology so that the module's datasheet characteristics are sufficient to model the PV arrays accurately. In the case of utility-scale PV systems, detailed knowledge of the module's electrical and optical properties of empirical data may be limited, and building a model is only possible by recurring to the datasheet information. A complex model that requires more detailed information may not be feasible in such cases, and a simpler model that relies on fewer input parameters is more appropriate. The single-diode model seems appropriate for this use case, given the excellent trade-off between complexity and accuracy.

\subsection{The five-parameter model}

Figure \ref{fig:onediodedraw} presents the single-diode model representation of the photovoltaic module. According to the five-parameter model, the unknown parameters are determined by fitting the model to experimental data or using data from the PV module's datasheet. The single-diode model can predict the PV module's performance under a wide range of operating conditions while maintaining reasonable accuracy. However, remembering that the single-diode model is a simplified representation of the PV module, it will have poor accuracy under certain situations compared to the more representative two-diode model \cite{Godina2017}.

\begin{figure}[H]
    \centering
    \includegraphics[width=10cm]{figures/chapter2/onediode.drawio.pdf} \caption{Single-diode model for photovoltaic modules.}
    \label{fig:onediodedraw}
\end{figure}

\section{Literature on Fault Detection and Classification for Photovoltaic Systems}

The parent field of fault detection is anomaly detection (also known as outlier detection), a highly studied subject in the scope of statistics \cite{Prasad2009}, applied in many scientific areas. Classification is also a well-studied subject in this field, with applications in numerous scientific contexts, from medical diagnosis to airport safety \cite{classification}. Consequently, adaptations of generic tools and ad hoc methodologies have originated to aid in solving fault detection and classification problems in photovoltaics.

According to \cite{AIPV}, the tools dedicated to PV fault detection and state estimation mostly come from mathematical/statistical methodologies, machine learning, and deep learning applications. Regarding the three general problem-solving principles mentioned before, it's known that machine learning and deep learning are the most popular and successful ones for recent applications that ought to solve complex problems. However, this categorization is somewhat limited, with contemporary literature suggesting an abundance of developed methodologies from different backgrounds, thoroughly reviewed in \cite{Hong2022} and \cite{Livera2019}. In \cite{Hong2022}, the authors consider two principal fault detection and classification algorithm branches: image-based and electrical-based; while \cite{Livera2019} also distinguishes numerical-based techniques. Image-based refers to aerial or visual capture of the PV array by photography and thermal imaging, commonly used along with artificial intelligence algorithms for assessing the photovoltaic module's state. Although the contribution and importance of such methods are appreciable, this work will mainly focus on the electrical-based and numerical-based ones, as the use case of the developed tool is bound to this type of data.

Categorizing methodologies becomes fuzzy, considering that some literature mixes physical behavior models with machine learning, statistics, and signal processing. Figure \ref{fig:techniques} is an attempt to present a structure inspired by the review made by \cite{Hong2022}, \cite{Livera2019}, and this work's, with a focus on the more relevant techniques (for this work's scope). Hybrid models are ubiquitous since combining robust statistical, signal processing, ML, or DL models and PV's electrical characterization can achieve more remarkable results. Hence, a better representation than figure \ref{fig:techniques} would be an incomprehensible mesh of connections representing the permutations between category aggregation.
 
\begin{figure}[h]
    \centering
    \includegraphics[width=15cm]{figures/chapter2/techniques.drawio.pdf} \caption{Representation of some of the methodologies employed in fault detection for PV systems.}
    \label{fig:techniques}
\end{figure}

To not wander in the literature, there must be a decision on which methodologies to revise. The developed tool in this work must meet certain real-life constraints, such as data availability, frequency, accuracy, PV system configuration, and context. Therefore, the (qualitative) potential estimation for each methodology will be based on the capability of adapting the proposed algorithms to the same expected restrictions. This evaluation process confines the methodology review to emphasize the ones thought to be most capable of implementation in a real scenario. Therefore, the following sections will not cover an extensive literature review, as it is not intended to repeat the works of \cite{Hong2022} and \cite{Livera2019}, only presenting interesting or adequate methodologies related to this work's scope.


\subsection{Statistical and Signal Processing Algorithms}

Statistical methodologies look into historical data to find the characteristics of how samples relate to the population (interpolation). These methodologies yield good results in case studies of PV farms that have been logging data for a considerable time, suffering in the cases that do not. Therefore, they are limited in that it is required to have curated data sets of historical significance for relevant features of the studied systems.

The literature on statistical and signal processing fault detection algorithms for PV is mostly quite dated (\cite{Buddha2012}, \cite{Zhao2014}, \cite{Vergura2008}), given that more recent machine learning methods have become increasingly attractive in this matter. Nonetheless, anomaly (or outlier) detection statistical algorithms can be used for fault detection in PV systems by identifying unusual patterns or deviations from normal behavior in the data collected from the PV system. Distance-based methods, such as the Euclidean, Mahalanobis, and MCD-based distances \cite{Braun2011}, may be adequate. Although simple, these techniques might only work for detecting outliers in the context of PV systems if they are scale-invariant (due to the different magnitude in the system's state variables) and resilient to outlier contamination (which only MCD-based distance is capable of). In \cite{Vergura2008}, the authors applied Analysis of Variance (ANOVA) and Kruskal-Wallis test for inverter failure detection, with the only downside of only being able to identify outliers in a sub-array resolution, i.e., not for specific string or module failures.

Some algorithms consider incoming data from PV systems as signals, allowing the adaptation of signal processing theory to develop ad hoc algorithms. Coming up with a relatively simple algorithm, the authors in \cite{Iles2021} propose a power-based fault detection method that only requires delayed samples of the PV array's power output and a threshold. Its reasoning is that since the power output of PV systems can't vary beyond a given point, considering a very short-term period (milliseconds), significant perturbations in this variable can be associated with faults. Although the simplicity and ease of implementation, it's clear that the success of this method requires feeding the algorithm with relatively high-frequency data, which would only be feasible on-site (and with specialized monitoring equipment).

In \cite{Fan2020}, the authors successfully formulated a graph signal processing algorithm for fault classification that yields increasingly better results when there is a considerable amount of labeled data, although its training is only semi-supervised. The results outperformed other standard machine learning methods for the same training data, given 30\% or more of labeled data. On another note, the data utilized came from the PVWatts \cite{Dobos2013} dataset, and the PV system is on a small scale (ASU testing facility \cite{Rao2016}) possessing a monitoring density and capability that can be considered irrealistic for utility-scale. This same data source is present in many other reviewed works.

The authors in \cite{Katoch2018} displayed another excellent use for graph theory, although not specifically for fault detection: they implemented a consensus-based distributed approach to minimize the impact of noise in acquired data from the PV array. By formulating a data propagation algorithm that resulted in measurement convergence, they achieved higher accuracy for state estimation.

With both graph theory-based algorithm proposals, this field sparks interest in its usage for the upcoming formulated methodology, given that it would be desirable to achieve an algorithm that features fault detection alongside data consensus.


\subsection{Machine Learning Algorithms} \label{subsec:machinelearning}

Machine learning is the trending way of solving increasingly complex and non-linear problems, as neural networks (or other learning structures) can better model complex, non-trivial, and nonlinear relations between data. Still, they are as good as the training data, with many designs requiring a lot of representative learning examples to achieve good results. Their output can also be very obfuscated (depending on the technique), meaning that many methods do not allow a direct interpretation of the relationship between inputs and outputs. This "black-box" characteristic, specifically of neural networks, is considered a disadvantage. Besides, extrapolating data remains a challenge when classically using these structures. Still, they have immense applications for PV systems, from MPP (Max Power Point) estimation to power forecasting, soiling, and fault prediction.

In \cite{Kumari2022}, an ANN is utilized to classify short circuit and hot spot faults. This algorithm achieved an outstanding 98.4\% classification accuracy, yet the data was simulated in \textit{MatLab/Simulink} and only considered two classes of faults. Because the inputs were the variation of voltage and current ($\frac{dV}{dt},\frac{dI}{dt}$), the algorithm required data sampling with relatively high frequency (>5Hz). The present work will not regard such methodologies as background for the upcoming tool since requiring high-frequency simulated data while covering only two fault types is quite far from a real utility-scale PV system scenario.

The trend of utilizing simulated data (sometimes without even added noise) has been a target of criticism in \cite{Aziz2020}. Accordingly, this work also emphasizes that the literature shows many proposed ML (and other types of) techniques that fall into this concept, which makes selecting appropriate methodologies to base future work on a challenging task.

The proposed ANN solution in \cite{Rao2019} is remarkable by the diversity of fault classification achieved: STC, short circuit, varying temperature, partial shading, complete shading, degraded modules, ground fault, and arc fault. It presents one of the most fault class coverage with high accuracy, considering the literature that utilizes synthetic noiseless data. Hence, the cyber-physical conceptualization and data preprocessing (clustering) demonstrated can be admired, but not forgetting that validation data came from a relatively unrealistic setting. 

In \cite{Rao2021}, there is a captivating proposal of utilizing an autoencoder and pruned neural network to separate the tasks of detecting and classifying faults, which resulted in one of the most performant ML approaches in the literature. The algorithm classifies five states: degraded, shaded, soiled, short circuit, and STC, utilizing nine inputs representing voltage, current, power, and irradiance available from the MPPT, datasheet, or meteorological sources. While the neural network pruning adds complexity, it resulted in a better generalized and lighter-weight trained model suitable for faster detection times. Even though using data from a small-scale PV system, the presented algorithm and its assumptions may make it possible to adapt and implement in an industrial scenario.

On the note of performance, the work in \cite{Kilic2020} proposes a sparse representation classifier (SRC) that evaluates if the system has line-to-line or line-to-ground faults for varying operating conditions. Although a drop in accuracy occurred for extreme circumstances, it is impressive that the algorithm identifies faults in such varied operating conditions: 10 to 50 degrees ambient temperature, 200 to 1000 $W/m^2$ irradiance, 10 to 60 \% of mismatch, and 0 to 25 $\Omega$ of fault resistance. The feature extraction step was also very impressive, which could be a determining factor in the method's performance. Unfortunately, this work also does not validate results with experimental data and only uses simulation as a source. However, the demonstrated computational performance, both in terms of training cost and utilization speed, its usage without the need for training for parameter tuning, the straightforward implementation, and consistent convergence, suggests the potential for this alternative in the face of other ML methodologies. The authors also emphasize that sparse representation might be utilized alongside different learning algorithms for classification, opening the door to many possible future implementations.

An exciting yet far-fetched proposal was made in \cite{Uehara2021}, where a quantum neural network (QNN) is formulated for PV fault classification. The QNN was trained for predicting just two scenarios: faulty or standard, but required up to four days of training, resulting in 93.89\% accuracy. For comparison, the classical ANN took twenty seconds to train and achieved 95.39\% accuracy. Although the methodology showcases the potential of quantum computing for this field, its preliminary results still distance itself from the traditional methods.

An abundance of ML methods have been tested and reviewed in this field (\cite{Hong2022},\cite{Livera2019}), utilizing structures such as SVM, KNN, RF, etc. Nonetheless, the results of \cite{Rao2021}-\cite{Kilic2020} sparked the most interest in this work's scope.

\subsection{Deep Learning Algorithms}

The field of deep learning is a branch of machine learning, with the term "deep" referring to amplified machine learning structures that ought to understand data patterns through more complex and intertwined artificial neuron connections. A simple example of a deep learning model would be the design of an artificial neural network with multiple hidden layers (DNN), with the intuition that each of these "extra" layers achieves feature/pattern recognition in a cascade. Other DL structures include the LSTM, CNN, and RBFNN. They have been explored alongside classical machine learning techniques for PV fault detection, although the known disadvantage is a usually high computational cost and relatively tricky implementation. These techniques are typically applied to image-based solutions \cite{termoreview} since they require classification based on 2D data from various image acquisition equipment \cite{termo}, \cite{dlpv}. Given the 1D characteristic of raw electrical data, not much literature considers these techniques for fault detection, as it implies an extra step of increasing dimensionality. However, there are some promising results in doing so \cite{Aziz2020}.

In \cite{Aziz2020}, not only is a DL technique presented for fault detection and classification, but there is also the best attempt at comparative evaluation against other methodologies. As already mentioned, this work exposes that much of the literature presents results solely based on particular datasets comprising simulated noiseless data, which invalidates any significant quantitative comparison. Consequently, a CNN model based on the pre-trained AlexNet \cite{Krizhevsky2012} is used both for classification or feature extraction, possibly allied with a classical ML model for classification in the latter. The classified faults were arc fault, partial shading, open circuit, and short circuit. While the experiments utilized simulation data, adding noise and an abundance of heterogenous operating conditions better resembled a real scenario. Considering the same noisy data, other tested methodologies present 22-70\% average accuracies, with the proposed fine-tuned AlexNet CNN reaching a maximum of 70.45\%. This work presents one of the best benchmarks in the literature, with decent coverage of other state-of-the-art ML and DL algorithms, while demonstrating the most realistic results and a sophisticated methodology proposal.

\subsection{Proposed method's scope}

While classical fault detection resides in the synchronous and direct evaluation of state and climate variables, realistic industrial scenarios can have data from various types, sources, and acquisition rates. It's also important to realize that monitoring equipment can register erroneous information, and current communication technology is also susceptible to delays and data loss. With this in mind, recent developments in the intelligent composition of deep learning structures aligned with graph theory spark some interest in their application to this field, such as the new deep learning technique named Cell Complex Neural Networks \cite{Hajij2020}. The motivation for choosing such a structure comes from its data propagation and consensus capability. The propagation techniques utilized in a CXN appeal to graph theory, dividing a system into other subsystems and components (nodes, also called cells in \cite{Hajij2020}) that share information. Even if the direct application of this structure might not be feasible or grant better results in the context of fault detection, its modification to meet the scope's needs could result in a robust and efficient solution. Further investigation of this state-of-the-art tool will unroll throughout the development of this work in an attempt to adapt this knowledge to the PV fault detection field.

According to the reviewed methodologies, the proposed tool should pertain to the DL or the hybrid category since, while having a central component of DL, it may also require modeling the PV system's components. The intention of proposing such a novel approach is to contribute to the deep learning methodology ecosystem, explicitly formulated for electrical-based PV fault detection and classification. As mentioned before, it aims at an asynchronous and online application, which differs from most current methods and presents a novel DL paradigm considering current knowledge. It is also desired that this work brings a comprehensive benchmark between popular methods (likewise \cite{Aziz2020}), utilizing a richer dataset with samples from tangible utility-scale PV assets, allowing accuracy assessment in a realistic scenario.