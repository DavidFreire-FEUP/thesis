\chapter{Bin Packing Problem} 
\label{chap:literature review}

The bin packing problem (BPP), as defined in the literature, is a combinatorial
optimization problem that has been extensively studied since the thirties. As
specified in \cite{wascher2007improved}, there are several categories with which
to define a BPP. The problem at hand is the one of a Single Bin Size Bin Packing
Problem (SBSBPP). As described in \cite{delorme2016bin}, an informal definition
of this BPP, is: provided there are n items, each with a given weight \( w_j  \,
(j = 1, ..., n) \) and an unlimited amount of bins with equal capacity \( C \),
the goal is to arrange the n items in such a way that the capacity of each bin
is not exceeded, and to determine the minimum amount of bins required to hold
the $n$ items.

Formally, the problem can be summarized as the following optimization problem:
\begin{alignat}{3}
    &\min       
        &&\sum_{i \in B} y_i 
            && \\
    &\text{subject to} \quad
        && \sum_{j \in L} w_j \cdot x_{ij} \leq C \cdot y_i \quad      
            && \forall \; i \in B \label{opt:c1} \\
    &   && \sum_{i \in B} x_{ij} = 1, \quad                 
            && \forall \; j \in L \label{opt:c2} \\
    &   && y_i \in \{0, 1\}                                 
            && \forall \; i \in B \\
    &   && x_{ij} \in \{0,1\}                               
            && \forall \; i \in B, j \in L
\end{alignat}
where set $B$ corresponds to the available bins, and $L$ as the list of items to
be arranged into different bins. Decision variable $x_{ij}$ and $y_i$, indicate
whether an item $j$ is packed in the bin $i$, and  whether bin $i$ is used
respectively. 

As for the Constraints, \ref{opt:c1} assures that the sum of the items in a bin,
does not exceed it's capacity, and \ref{opt:c2} makes sure that every item is
assigned a bin. 

It is also common to represent the bin-packing problem in its normalized
version, where all the weights are down-scaled by the total capacity of a bin,
and the capacity of each bin is $1$. This implies modifying Constraint
\ref{opt:c1} into:

\begin{equation}
    \sum_{j \in L} \hat w_j \cdot x_{ij} \leq y_i \quad \forall \; i \in B,
\end{equation}
where \(\hat w_j\) represents the normalized weight of an item ($w_j/C$).
Throughout the following anaylsis, both "weight" and "size" will be used
interchangeably, and the normalized BPP model will be employed.

\section{Linear Programming}

When interested in achieving the optimal solution, the textbook implementation
of the BPP model presented by Equation \ref{BPP model} is not computationally
efficient. A common approach is to study the Upper and Lower Bounds of the
Algorithm, and to add valid constraints to restrict the search space.

In \cite{delorme2016bin}, there is an extensive review on the state-of-the-art
algorithms that have been developed to solve the ILP problem, comparing several
models and their efficiency when solving the same set of problems. 

Within the context of the problem presented in chapter 1, provided two time
instants $t_1$ (the last instant at which the item assignment was computed) and
$t_2$ (the current time instant at which the assignment is to be re-computed),
given that the items' sizes change between $t_1$ and $t_2$, it is possible that
the configuration computed in $t_1$ no longer complies with Constraint
\ref{opt:c1}. Considering the NP-hard nature of the problem, and the fact that the
ILP problem is only interested in minimizing the amount of bins disregarding an
item's previous assignment, this thesis will give more emphasis to the
Approximation Algorithms that provide more control over the item reassignment
problem, and provide a solution within the strict time constraints imposed by
the real-time requirements presented in Chapter \ref{chap:intro}.

\section{Approximation Algorithms and Heuristics}
\label{section:AA}

A method is presented to classify the BPP problem \cite{coffman2013bin}, which
will be used throughout this section. During an algorithm's execution, a bin can
find itself either \textit{open} or \textit{closed}. In the former, the bin can
still be used to add additional items, whereas in the latter, it is no longer
available and has already been used.  

In the literature, it is common to present a parameter $\alpha$ which
indicates the size limit of the weights within the list $L$, which varies
between $]0, 1]$. This thesis studies the problem where the item's size is
simply limited to the size of the bin, and therefore $\alpha = 1$.

$A(L)$ denotes the amount of bins a certain algorithm makes use of for a
configuration of items $L$. $OPT(L)$ is used to represent the amount of bins
required to achieve the optimal solution. Defining $\Omega$ as the set of all
possible lists, each with a different arrangement of their items, given a
performance ratio defined by 
\begin{equation}
    R_A (k) = \sup_{L \in \Omega} \left \{ \frac{A(L)}{k} : k = OPT(L) \right \},
\end{equation}
the Asymptotic Performance Ratio (APR) of an algorithm $A$ ($R_A^\infty$) is
defined as
\begin{equation}
    R_A^\infty = \limsup_{k \to \infty} R_A(k).
\end{equation}

Despite the mulititude of classes presented in \cite{coffman2013bin}, the only
classes of problems that are of interest for this thesis are: 
\begin{enumerate}
    \item online - Algorithms that have no holistic view over any other item in
        the list except the one it is currently assigning a bin to. It is also a
        requirement that an item is assigned a bin as soon as it is analyzed.
    \item bounded-space - The amount of bins open at a single instance is
        limited by a constant provided prior to its execution.
    \item offline - The algorithm is aware of all the items in the list before
        assigning bins to each item, and so their ordering does not directly
        impact the outcome as it is not necessary to respect the list's initial
        order.
\end{enumerate}

\subsection{Online Algorithms}

Whenever an item is analyzed it has to be assigned a bin. This also implies that
the bin in which the current item will be inserted is a function of the weights
of all the preceding items in the list.

This section analyzes Any fit, Almost any fit, bounded-space and reservation
technique algorithms.  Throughout the following sections, when \textit{current
item} is mentioned, it is referring to the next item to assign a bin to. After
assignment, the item that follows it in the list, is then considered the
\textit{current item}.

\textbf{Next Fit (NF)} is one of the simplest algorithms developed to solve the
bin packing problem heuristically, and it consists in the following procedure.
The current bin is considered to be the last non-empty bin opened. If the
current bin has space for the item then the item is inserted. Otherwise, the
current bin is closed, a new one is opened, and the item is inserted. The next
item on the list is then considered the current item.

With regards to time complexity, NF is $O(n)$. And because there is only one
open bin at a time, this is a bounded-space algorithm with $k = 1$. As proven in
\cite{johnson1973near}, 
\begin{equation}
    R_{NF}^\infty = 2.
\end{equation}

\textbf{Worst Fit (WF)}: For each element, it looks for the open bin that has
the most space where it fits, and inserts the item in that bin. If there is no
open bin that can hold the item, then a new bin is opened, and the item is
inserted. Having no closing procedure, this algorithm is unbounded, and it does
not run in linear time.

Although it is expected that this algorithm would perform better than the NF, in
the worst-case performance analysis it does not gain any benefits from not
closing its bins, as stated in \cite{man1996approximation}
\begin{equation}
    R_{WF}^\infty = R_{NF}^\infty.
\end{equation}

\textbf{First Fit (FF)} searches each open bin starting from the lowest index,
and the first bin that has the capacity to fit the item, is where the item is
inserted. If there is no open bin where the item can be inserted, then a new bin
is opened where the item is inserted. In \cite{johnson1974worst} it is shown
that
\begin{equation}
    R_{FF}^\infty = \frac{17}{10}.
\end{equation}

There is a more general class of algorithms presented as \textit{Any Fit} ($AF$)
in \cite{johnson1974fast}. This class' constraint is that if $BIN_j$ is empty,
then no item will be inserted into this bin if there is any bin to the left of
$BIN_j$ that has the capacity to contain the item. In the same paper it is also
shown that no algorithm that fits this constraint can perform worst than the WF,
nor can it perform better than the FF, always with respect to the asymptotic
performance ratio. Then
\begin{equation}
    R_{FF}^\infty \leq R_A^\infty \leq R_{WF}^\infty, \; \forall \; A \in AF
\end{equation}

\textbf{Best fit (BF)}: Attempts to fit the item into any of the open bins where
it fits the tightest. If it doesn't fit in any, then a new bin is created and
assigned to the item. As can be seen, it is similar to the worst fit, with
exception to the fitting condition, differing only in the fact that the item is
inserted where it fits the tightest, and not where it leaves most slack.

As it happens, BF also belongs to the $AF$ class, and it performs as well as the
First Fit
\begin{equation}
    R_{BF}^\infty = R_{FF}^\infty.
\end{equation}

In \cite{johnson1974fast}, another class of algorithms presented, which is
also a subclass of $AF$, is \textit{Almost Any Fit} ($AAF$). This class has as
constraints: \textit{If $BIN_j$ is the first empty bin, and $BIN_k$ is the bin
that has the most slack, where $k < j$, then the current item is not inserted
into $k$ if it fits into any bin to the left of k.} One of the properties proven
in the same paper, is that 
\begin{equation}
    R_A^\infty = R_{FF}^\infty, \; \forall \; A \in AAF,
\end{equation}
which is to say that any algorithm that fits the previous constraints, have the
same APR as the FF algorithm. Focusing on the constraints, it is clear to see
how BF and FF belong to this class. 

Due to the constraints presented in the $AAF$ class, an improvement for the WF
algorithm arises wherein the current item is inserted in the second bin with
most space, instead of the first. This algorithm is the Almost Worst Fit (AWF),
and with this simple change, now belongs to the AAF class, having as APR
\begin{equation}
    R_{AWF}^\infty = R_{FF}^\infty = \frac{17}{10}.
\end{equation}

\subsection{Bounded-space} 

Bounded-space algorithms have a predefined limit on the amount of bins that are
allowed to be open at a given instance, which will be defined as $k$. It is also
a subclass of the online algorithms.

An example of a bounded-space algorithm is NF, as it never has more than a
single open bin at a given instant. The other presented online algorithms can
also be adapted into a bounded space algorithm, simply by specifying a procedure
as to which bin to close before exceeding the limit.

A bounded-space algorithm can be defined via their packing and closing rules
\cite{coffman2013bin}. A class that derives from rules based on the FF and BF
can be defined in the following manner: 
\begin{itemize}
    \item Packing - When packing an item into one of the available open bins,
        the selected bin either follows a FF or BF approach.
    \item Closing - When choosing which bin to close, if following the FF
        approach, then the bin with the lowest index is closed. If following the
        BF it's the bin that is filled the most that is selected as the one to
        be closed.
\end{itemize}

The notation for this class of algorithms is $AXY_k$, where X represents the
packing rule, and assumes the letters $F$ or $B$, and $Y$ which can either be an
$F$ or a $B$, refers to the closing rule. The $k$ represents the maximum amount
of open bins allowed.

\textbf{Next-k-fit} applies both packing and closing rules based on the first
fit algorithm, and as expected when $k \to \infty$ it approximates the FF
algorithm, having as APR $17/10$. As shown in \cite{mao1993tight}

\begin{equation}
    R_{AFF_k}^\infty = \frac{17}{10} + \frac{3}{10k - 10}, \quad \forall \; k \geq 2.
\end{equation}

\textbf{Best-k-fit}, which is also known as the $ABF_k$ algorithm, has been
proven in \cite{mao1993besk} to have
\begin{equation}
    R_{ABF_k}^\infty = \frac{17}{10} + \frac{3}{10k}, \quad  \forall \; k \geq 2.
\end{equation}

As for $AFB_k$, \cite{zhang1994tight} showed that this algorithm has
\begin{equation}
    R_{AFB_k}^\infty = R_{AFF_k}^\infty, \quad  \forall \; k \geq 2.
\end{equation}

For the last possible combination of this class of algorithms, $ABB_k$ has been
proven in \cite{csirik2001bounded} to have 
\begin{equation}
    R_{ABB_k}^\infty = \frac{17}{10},  \quad \forall \; k \geq 2
\end{equation}
which surprisingly indicates that the value of $k$ (as long as it's bigger than
two) has no effect on the APR metric of this algorithm, contrary to all the
previous algorithms.

The next algorithms make use of a reservation technique that proved to be very
useful to break the lower bound of the APR of the Any Fit class of algorithms.
The first algorithm to be developed of this type was the Harmonic-Fit ($HF_k$)
which makes use of k to split the sizes of items into $k$ different intervals.

$I_j$ denotes the interval of sizes with index j, and is within the range 
\begin{equation}
    \left (\frac{1}{j+1}, \frac{1}{j} \right], \quad \forall \; j \in \{1, ..., k-1\}.
\end{equation}
$I_k$ is defined as the interval from $(0, 1/k]$.

$B_j$ is used to classify the bin type which is responsible for holding items of
type $I_j$.

It is shown in \cite{lee1985simple} that for $HF_k \forall k \geq 7$ the
algorithm already performs better than the other online bin packing algorithms
presented, with regards to the APR.

Posterior to this technique being presented, it was clear that the reservation
technique could be a good approach to improve the performance of the Any-Fit
algorithms, and as such, several other algorithms have been created around this
technique, that achieve even better APR's than HF, but because these techniques
all involve assigning item types, based on their size, to their respective bin
type, this approach is not applicable to the problem, as the control over item
rebalancing is reduced, which will further be described in the following
chapter.

\subsection{Offline Algorithms}

Comparing with the previous section, offline algorithms have the added benefit
that all items are known prior to its execution. As long as the set of items
remains the same, items can be grouped, sorted or anything that might be
convenient for the algorithm that is going to execute over the list of items. 

It is important to note that as soon as an algorithm chooses to sort the list of
items, it automatically implies that the algorithm no longer runs in linear-time
as the sorting algorithm would have a time complexity of $O(n log(n))$.

Most of the Any Fit algorithms, perform best if the list of items is sorted in a
non-increasing order. The following three algorithms remain the same as for how
the packing is done when analyzing the list of items, with the exception that
before running the algorithm, the list is sorted.

Provided the list is sorted in the aforementioned manner, the \textbf{Next Fit
Decreasing (NFD)} has a considerable improvement in terms of it's worst-case
performance, and performs slightly better than FF and BF, as presented by
\cite{baker1981tight}
\begin{equation}
    R_{NFD}^\infty \approx 1.6910.
\end{equation}

As can be seen in \cite{johnson1974worst}, \textbf{Best Fit Decreasing (BFD)}
and \textbf{First Fit Decreasing (FFD)} improve the APR metric with presorting
compared to their online versions of the same algorithm 
\begin{equation}
    R_{BFD}^\infty = R_{FFD}^\infty = \frac{11}{9}.
\end{equation}

Another algorithm which is worth mentioning within this class of offline
algorithms is the \textbf{Modified First Fit Decreasing (MFFD)}. As shown in
\cite{johnson19857160}, The APR is
\begin{equation}
    R_{MFFD}^\infty = \frac{71}{60},
\end{equation}
which is achieved by initially presorting the items in a decreasing manner and
grouping each item into seven distinct groups based on the item's size. 

\begin{table}[H]
\caption{Item size interval which guides the grouping for the MFFD algorithm.}
\begin{center}
\begin{tabular}{ |c|c| } 
    \hline
    Group & Item size interval \\ 
    \hline
    A & $\left( 1/2, 1 \right]$ \\ 
    B & $\left( 1/3, 1/2 \right]$ \\ 
    C & $\left( 1/4, 1/3 \right]$ \\ 
    D & $\left( 1/5, 1/4 \right]$ \\ 
    E & $\left( 1/6, 1/5 \right]$ \\ 
    F & $\left( 11/71, 1/6 \right]$ \\ 
    G & $\left( 0, 11/71 \right]$ \\ 
\hline
\end{tabular}
\end{center}
\end{table}

After doing so, the algorithm then performs the following five steps: 
\begin{enumerate}
    \item For each item that belongs to group A, from biggest to smallest,
        assign it a bin with the same index as the item has within it's group.
        When this process terminates, there are as many bins as items in group A
        and the bins created are $B_1, ..., B_{|A|}$.
    \item Iterating over the existing bins from left to right, for each bin, if
        any item in group $B$ fits in the bin, insert the biggest such item in
        the current bin.
    \item Iterating through the list of bins from right to left, for each bin,
        if the current bin contains an item that belongs to group $B$, do
        nothing. If the two smallest items in $C \cup D \cup E$ do not fit, do
        nothing. Otherwise insert the smallest unpacked items from $C \cup D
        \cup E$ combined with the largest item from $C \cup D \cup E$ that will
        fit.
    \item iterate over the list of bins from left to right, and for each bin if
        any unpacked item fits, insert the largest such item, repeating until no
        unpacked item fits.
    \item Lastly, the remaining items that did not fit in bins $B_1, ...,
        B_{|A|}$ are inserted in an FFD fashion starting with a new bin
        $B_{|A|+1}$. 
\end{enumerate}

\section{Bin Packing Applications}

There are several applications where the BPP is used to provide with an optimal or
sub-optimal solution. In \cite{baldi2019generalized}, the problem is modeled as
a BPP to solve the last mile delivery problem a courier usually faces when
having to decide which transportation companies are to be used to deliver the
parcels to the customers. In this case, a Transportation Company (TC) represents
a type of bin, and a vehicle of a TC represents a bin instance. The problem is
formulated as a generalization of the BPP, considering a profit value is
incorporated into the cost function which is based on the bin-type each item
ends up assigned to, which in the problem's context, this profit represents the TC's
QoS. As such, the objective function aims to minimize the net cost related to
the vehicle instances used to allocate the parcels and the profits associated
with the parcel assignment.

Another application where the BPP proves itself useful is in the The Virtual
Machine Placement (VMP) problem, which has gained more attention due to
increasing use of cloud providers to support companies' technological
infrastructure. Usually the problem can be generalized to a set of tasks
(Virtual Machines), each having to be assigned one of the cloud provider's
physical machines while attempting to minimize the operational cost. A variation
of the BPP is usually presented in this case where the items are considered
ephemeral and add another temporal dimension since they have associated a start
time and a duration during which the task has to be ran uninterrupted. 

In \cite{de2016temporal} the tasks have an equal start time and the aim is to
minimize the unused capacity of a physical machine over time, and therefore the
optimal solution for the BPP does not necessarily provide the optimal solution
in their Temporal Variant of the BPP (TBPP). In addition to the temporal model
presented by \cite{de2016temporal}, in \cite{aydin2020multi} the tasks may have
differing start times, and the objective function presented is more in line with
the traditional BPP where the number of physical machines required is minimized.
Moreover, due to the increased energy costs related to machine fire-ups
(switching a machine off when it is not in use and back on again when required),
the authors also attempt to minimize the number of fire-ups, making their
problem multi-objective.

As presented in \cite{furini2018matheuristics}, provided a set of items $N =
\{1,2,...,n\}$, a weight $w_i \; \forall i \in N$, and a start and end time
$[s_i, t_i[$, the set of items which have to be executing at a time instant
$s_j$ is $S_j = \{n \in N: s_n \leq s_j \; and \; t_n > s_j\}$. This allows the
definition of the non-dominated set $T = \{n \in N: S_n \not\subseteq S_k\, \
\forall \ k \in N\}$ which discretizes the time dimension into several time
instants. Due to the stateful nature of the problem that allows a task to be in
more than one $S_j$, which implies an item may have an assignment from a
previous time instant, and since an item cannot be rebalanced, an optimal
solution for the $BPP(t)$ is not the same as an optimal solution for $TBPP(t)$.

\section{Conclusion}

Although the temporal aspect of the Bin Packing Problem has already been
reviewed, there is a gap when it comes to the possibility of rebalancing the
items between the different bin instances. Existing solutions consider a bin
packing model where the items' sizes do not change with time, and the assignment
is persistent and ephemeral, i.e., the items require an uninterrupted assignment
for a specified duration. Modeling the BPP with items' sizes and consumer
assignments that vary over time, is a new variation that has not been studied,
and will be one of the contributions of this work. 

In view of the fact that the existing approximation algorithms were developed to
solve a single iteration of the BPP, these have to be adapted to the problem at
hand, which requires solving a new iteration of the BPP given an already
existing assignment. Despite the fact that the adaptation presented in Section
\ref{sub:reassign} improves these algorithms' performance with respect to the
rebalance cost, there is still room for improvement. The Modified Algorithms
presented in Section \ref{subsub:modified_any_fit} are developed to solve this
variation of the BPP, while taking the rebalance cost into account.

% \section{Distributed Consumer Performance}

% \section{Filters}

%\section{Summary}
