\chapter{CellTAN Application} \label{chap:chap5}

\dots

\section{Case study}

Our test case is based on two neighboring inverters of the same PV farm, with common satellite data.

\subsection{Data analysis}

% analise e limpeza de dados. Que cenários queremos detetar, o que é que podem ser
\dots

\subsection{Photovoltaic Plugin}

% formulação do algoritmo para deteçao de falhas em inversores
\dots

\subsection{CellTAN Configuration}

% config das células, como vai correr, como se simula o tempo
\dots

\subsection{Simulation and Results}

\dots

\subsection{Scaling up}

\begin{figure}[h!]
    \centering
    \includegraphics[width=\linewidth]{figures/chapter4/cell/celltan.pdf}
    \caption{Ilustrative overview of a CellTAN.}
    \label{fig:celltan}
\end{figure}

Figure \ref{fig:celltan} illustrates a simple CellTAN network overview. Regarding the \textbf{Hub}, one might infer (correctly) that a central component breaks the non-centralized paradigm. Nevertheless, it is present to solve some real-life implementation challenges and limitations, as will be further discussed in this chapter.
