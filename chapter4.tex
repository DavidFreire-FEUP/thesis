\chapter{CellTAN: Cellular Time Activation Networks} \label{chap:chap4}

This chapter proposes a new tool called CellTAN (Cellular Time Activation Networks) that can effectively address the challenges of distributed information systems characterized by time series data. CellTAN represents sparse yet interconnected components that function independently, cooperatively, and asynchronously. While initially designed for photovoltaic (PV) systems, the concepts of cells, connections, neighbors, time series, and uncertainty are universal and applicable to other fields such as biology, physics, and more.
Inspired by other effective mechanisms like GNNs, CXNs, and Weighted Cross-Connection Networks (WCCNs), CellTAN uses a graph-like structure to represent a network of components. This structure allows for decentralized knowledge and requires minimal resources for each component to operate in real time.
The subsequent sections of this chapter will explain the working of the "cell" and its interactions within the network, offering a comprehensive understanding of the tool.

\section{Glossary}

\begin{itemize}
    \item \textbf{Knowledge base}: refers to registered historical knowledge (samples) of time series variables.
    
    \item \textbf{Inputs}: a set of variables that define the state of a cell. Instantaneous values, represented by uniform fuzzy numbers.

    \item \textbf{Outputs}: similar to inputs, but obtained through certain computations.

    \item \textbf{Time decay}: a process associated with increased uncertainty of variables over time.
    
    \item \textbf{Time Activations}: a series of time intervals defined by start and end timestamps.

    \item \textbf{Hub}: the central component of the cell network, which facilitates its visualization, management, and expansion. It acts as the proxy agent between the cell's communication.
    
\end{itemize}

\section{The Cell}

Inside a cell there are artifacts (data) and processes.

\subsection{Cell's Core}

\subsubsection{Temporal similarity extraction}

Temporal similarity extraction is essential for identifying recurring patterns in time series data. It involves the identification of past instances where the current state is observed to extract useful information about the system's behavior over time. By identifying historical periods with similar states, temporal similarity extraction can help assess the current state or predict future trends. This technique is prized in finance, medicine, and environmental science, where accurate forecasting is critical for effective planning and management. However, the process of temporal similarity extraction can be challenging due to the complexity and size of time series data, as well as the need to accurately measure and compare similarity across different periods.

One approach to simplifying the process of temporal similarity extraction in multi-variate time series data is to use uniform fuzzy values to filter historical data. This technique involves assigning fuzzy membership values to each data point in the time series based on its similarity to the current state. The use of fuzzy values can capture the inherent uncertainty in time series data, making the approach robust to noisy or incomplete data. Moreover, uniform fuzzy values make filtering history trivial and easily streamlined/automated.


\paragraph{Self Similarity}

Having a knowledge base, the cell is capable of performing temporal similarity extraction with itself given its set of inputs.

\paragraph{Mutual Similarity}
Capacidade de estimação de estado intrínsica + extrínsica

\subsection{Connections and Trust}
O que é uma conexão, como se caracteriza, e que valores tem a ela associada
