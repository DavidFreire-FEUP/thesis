\chapter*{Resumo}
%\addcontentsline{toc}{chapter}{Resumo}

A proliferação de centrais fotovoltaicas de dimensão industrial levou à necessidade de métodos para detetar e classificar falhas nos seus componentes, sendo que estas que podem ter impactos económicos significativos. Neste trabalho, explora-se o estado da arte das ferramentas de deteção de falhas e estimação do estado aplicadas ao campo dos sistemas PV, com foco na compreensão do seu funcionamento, identificando-se pontos fortes e possíveis limitações. Relevar-se-á quais os métodos baseados em aprendizagem computacional mais utilizados. Ainda assim, reconhece-se o contributo dos diversos domínios para colmatar este tipo de problema, desde a teoria dos grafos a processamento de sinal, aprendizagem profunda e aprendizagem quântica. Efetuam-se comparações e propostas de melhoria aos algoritmos existentes, e desenvolvida uma nova abordagem para abordar o tema de deteção de falhas. Com a retrospeção das ferramentas contemporâneas de maior sucesso, e pela oferta de uma nova abordagem, o objetivo deste trabalho é fornecer aos operadores de instalações fotovoltaicas o aumento na fiabilidade e eficiência dos seus sistemas. Além disso, há a possibilidade de que a ferramenta desenvolvida seja aplicável para outros problemas de coesão de dados, impactando positivamente os diversos tipos de domínios de sistemas orientados a dados.


\chapter*{Abstract}
%\addcontentsline{toc}{chapter}{Abstract}
The rise of utility-scale photovoltaic power plants has increased the need for reliable fault detection methods to minimize economic losses caused by downtime. These methods utilize advanced monitoring and diagnostic techniques to identify anomalies, such as faulty panels or degraded components, which can lead to decreased power generation or system failure. By swiftly detecting and addressing faults, operators can minimize downtime, allocate resources efficiently, and optimize maintenance efforts. Continuous monitoring and analysis of crucial parameters enable early detection of anomalies, ensuring uninterrupted power generation. Effective fault detection methods enhance the reliability and resilience of photovoltaic power plants, reducing the risk of cascading failures and disruptions. As the renewable energy sector expands, ongoing development and improvement of fault detection techniques are crucial for maintaining the efficiency and profitability of utility-scale photovoltaic power plants.

We thoroughly evaluate fault detection tools' current status in photovoltaic (PV) systems. The intention is to understand how these tools work and identify their strengths and limitations. Based on an extensive literature review, we recognize that machine learning techniques are the foundation of state-of-the-art fault detection and classification algorithms. We also recognize the multidisciplinary nature of this field, with contributions from graph theory, signal processing, deep learning, and even emerging areas like quantum machine learning. By exploring these different approaches, we provide valuable insights into the advancements in fault detection and state estimation for PV systems, opening doors for further developments and improvements in this crucial research area.

Stemming from the acquired knowledge base, we propose a novel approach to tackle anomaly detection with a new paradigm. CellTAN (Cellular Time Activations Networks) is an asynchronous and distributed framework that leverages cell relationships to assess their state and identify anomalies. It boasts data privacy, component cooperation without data normalization, and independent operation.
We test it against industrial-scale PV farm data (samples of two inverters) to validate anomaly detection capability and correctly assess specific situations.


Our approach yields promising results in identifying inverter anomalies based on their cell relationship and underperformance scenarios. The effectiveness of our method is noteworthy as it does not rely on specific system information nor breaks data privacy, enabling its application to diverse components from different stakeholders. This characteristic makes it versatile and applicable to any time-series system, enhancing its usability and potential impact. By demonstrating its efficacy in detecting anomalies and accommodating various system configurations, our approach offers a valuable solution for fault detection in diverse contexts, such as wind farms, the electrical grid, or other physical systems.

By examining the most successful tools to date and offering a new solution, we contribute to improving the reliability and efficiency of PV systems. Furthermore, our generalistic approach contributes to anomaly detection for other fields.
