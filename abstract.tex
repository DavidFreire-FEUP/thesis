\chapter*{Resumo}
%\addcontentsline{toc}{chapter}{Resumo}

Este documento ilustra o formato a usar em dissertações na \Feup.
São dados exemplos de margens, cabeçalhos, títulos, paginação, estilos
de índices, etc. 
São ainda dados exemplos de formatação de citações, figuras e tabelas,
equações, referências cruzadas, lista de referências e índices.
Este documento não pretende exemplificar conteúdos a usar. 
É usado o \emph{Loren Ipsum} para preencher a dissertação.

Lorem ipsum dolor sit amet, consectetuer adipiscing elit. Etiam vitae
quam sed mauris auctor porttitor. Mauris porta sem vitae arcu sagittis
facilisis. Proin sodales risus sit amet arcu. Quisque eu pede eu elit
pulvinar porttitor. Maecenas dignissim tincidunt dui. Pellentesque
habitant morbi tristique senectus et netus et malesuada fames ac
turpis egestas. Donec non augue sit amet nulla gravida
rutrum. Vestibulum ante ipsum primis in faucibus orci luctus et
ultrices posuere cubilia Curae; Nunc at nunc. Etiam egestas. 

Donec malesuada pede eget nunc. Fusce porttitor felis eget mi mattis
vestibulum. Pellentesque faucibus. Cras adipiscing dolor quis
mi. Quisque sagittis, justo sed dapibus pharetra, lectus velit
tincidunt eros, ac fermentum nulla velit vel sapien. Vestibulum sem
mauris, hendrerit non, feugiat ac, varius ornare, lectus. Praesent
urna tellus, euismod in, hendrerit sit amet, pretium vitae,
nisi. Proin nisl sem, ultrices eget, faucibus a, feugiat non,
purus. Etiam mi tortor, convallis quis, pharetra ut, consectetuer eu,
orci. Vivamus aliquet. Aenean mollis fringilla erat. Vivamus mollis,
purus at pellentesque faucibus, sapien lorem eleifend quam, mollis
luctus mi purus in dui. Maecenas volutpat mauris eu lectus. Morbi vel
risus et dolor bibendum malesuada. Donec feugiat tristique erat. Nam
porta auctor mi. Nulla purus. Nam aliquam. 


\chapter*{Abstract}
%\addcontentsline{toc}{chapter}{Abstract}

The increase in photovoltaic power plants has led to the need for effective
methods for detecting and addressing component faults, which can have
significant economic impacts. In this work, we will explore the current state of
fault detection and state estimation tools in the field of PV systems, with a
focus on understanding how these tools work and identifying their strengths and
limitations. We will also propose improvements to existing approaches or develop
a novel approach to address this issue. By examining the most successful tools
to date and offering new solutions, we aim to help PV plant operators improve
the reliability and efficiency of their systems.