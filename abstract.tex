\chapter*{Resumo}
%\addcontentsline{toc}{chapter}{Resumo}

A proliferação de centrais fotovoltaicas leva à necessidade de métodos para detetar e classificar falhas nos seus componentes, sendo que estas que podem ter impactos económicos significativos. Neste trabalho, o estado da arte das ferramentas de deteção de falhas e estimação do estado aplicadas ao campo dos sistemas PV será explorado, com foco na compreensão do seu funcionamento, identificando-se pontos fortes e possíveis limitações. Serão propostas melhorias às abordagens existentes ou desenvolvida uma nova abordagem para abordar esta problemática. Com a inspeção das ferramentas mais bem sucedidas até à data e pela potencial oferta de uma nova abordagem, o objetivo deste trabalho é fornecer aos operadores de instalações fotovoltaicas um aumento na fiabilidade e eficiência dos seus sistemas.


\chapter*{Abstract}
%\addcontentsline{toc}{chapter}{Abstract}

The increase in photovoltaic power plants has led to the need for effective methods for detecting and classifying component faults, which can have significant economic impacts. In this work, we will explore the current state of fault detection and state estimation tools in the field of PV systems, with a focus on understanding how these tools work and identifying their strengths and limitations. We will also propose improvements to existing approaches or develop a novel approach to address this issue. By examining the most successful tools to date and offering new solutions, we aim to help PV plant operators improve the reliability and efficiency of their systems.