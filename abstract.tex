\chapter*{Resumo}
%\addcontentsline{toc}{chapter}{Resumo}

A proliferação de centrais fotovoltaicas de dimensão industrial levou à necessidade de métodos para detetar e classificar falhas nos seus componentes, sendo que estas que podem ter impactos económicos significativos. Neste trabalho, explora-se o estado da arte das ferramentas de deteção de falhas e estimação do estado aplicadas ao campo dos sistemas PV, com foco na compreensão do seu funcionamento, identificando-se pontos fortes e possíveis limitações. Relevar-se-á quais os métodos baseados em aprendizagem computacional mais utilizados. Ainda assim, reconhece-se o contributo dos diversos domínios para colmatar este tipo de problema, desde a teoria dos grafos a processamento de sinal, aprendizagem profunda e aprendizagem quântica. Efetuam-se comparações e propostas de melhoria aos algoritmos existentes, e desenvolvida uma nova abordagem para abordar o tema de deteção de falhas. Com a retrospeção das ferramentas contemporâneas de maior sucesso, e pela oferta de uma nova abordagem, o objetivo deste trabalho é fornecer aos operadores de instalações fotovoltaicas o aumento na fiabilidade e eficiência dos seus sistemas. Além disso, há a possibilidade de que a ferramenta desenvolvida seja aplicável para outros problemas de coesão de dados, impactando positivamente os diversos tipos de domínios de sistemas orientados a dados.


\chapter*{Abstract}
%\addcontentsline{toc}{chapter}{Abstract}

The increase in utility-scale photovoltaic power plants has led to the need for effective methods for detecting and classifying component faults, which can have significant economic impacts. This work assesses the current state of fault detection and state estimation tools in the field of PV systems, focusing on understanding how these tools work and identifying their strengths and limitations. It is seen that machine learning makes up the majority of state-of-the-art fault detection and classification algorithms. Still, many fields have contributed to this problem, from graph theory to signal processing, deep learning, and quantum machine learning. Consequently, this work compares and proposes improvements to existing approaches or a novel technique developed to address this issue. By examining the most successful tools to date and offering new solutions, the intention is to help PV plant operators improve the reliability and efficiency of their systems. The developed methodology is also expected to become a generalistic data cohesion algorithm, positively impacting other data-driven fields.