\chapter*{Resumo}
%\addcontentsline{toc}{chapter}{Resumo}

O aumento da capacidade fotovoltaica mundial leva à necessidade de métodos de deteção de falhas fiáveis para se minimizarem perdas financeiras devido a paragens de produção. Tais métodos recorrem a técnicas avançadas de monitorização e diagnóstico para identificar situações que possam levar à diminuição da produção energética ou à falha de sistema, tais como painéis ou outros componentes defeituosos/degradados. Caso haja detetação e resolução de falhas rapidamente, os operadores minimizam o tempo de inatividade do sistema, permitindo gerir os recursos de forma mais eficiente e otimizar esforços de manutenção. Com a expansão do sector das energias renováveis, o desenvolvimento e melhoria contínuos nas técnicas de deteção de falhas são cruciais para manter a eficiência e a rentabilidade das centrais fotovoltaicas.

Neste trabalho faz-se uma avaliação exaustiva do estado atual das ferramentas de deteção de falhas em sistemas fotovoltaicos (PV). Tenciona-se compreender o funcionamento destas ferramentas e identificar quais os seus pontos fortes e limitações. Com base na revisão bibliográfica, reconhece-se que as técnicas de aprendizagem computacional são a base dos algoritmos mais avançados de deteção e classificação de falhas. Reconhece-se também a natureza multidisciplinar deste domínio, com contributos da teoria de grafos, processamento de sinal, aprendizagem profunda e até de áreas emergentes como a aprendizagem computacional quântica. Ao explorar as diferentes abordagens, relata-se os avanços na deteção de falhas para sistemas fotovoltaicos, traçando o caminho para novas formulações e melhorias nesta área de investigação.

Partindo da base de conhecimento adquirida, propõe-se uma nova abordagem para deteção de anomalias, baseada num paradigma divergente à literatura. CellTAN (\textit{Cellular Time Activations Networks}) é uma estrutura assíncrona e distribuída que usa as relações entre células para avaliar o seu estado e identificar anomalias. Mantém a privacidade de dados, permite cooperação entre componentes sem normalização de dados e operação independente. As funcionalidades desta ferramenta transcendem a sua aplicação além dos sistemas fotovoltaicos, servindo como uma base para todo o género de deteção de anomalias em sistemas dinâmicos.
Testa-se esta abordagem com dados de um parque fotovoltaico de grande escala (amostra de dois inversores e um satélite) para validar a sua capacidade de deteção de anomalias e algumas situações específicas.

A nova abordagem demonstra resultados promissores, com identificação de anomalias no inversores baseada apenas nas relações celulares e cenários de perda de desempenho mapeados. A sua eficácia é significativa, uma vez que não depende de informações específicas do sistema nem quebra a privacidade dos dados, permitindo a sua aplicação a diversos tipos de componentes com diferentes donos. Estas caraterísticas tornam tal abordagem mais versátil e aplicável a qualquer sistema assente em séries temporais, aumentando o seu potencial impacto noutros ramos de conhecimento. Ao demonstrar a sua eficácia na deteção de anomalias e ao acomodar várias configurações do sistema, esta ferramenta oferece uma solução de valor acrescido para a deteção de falhas em diversos contextos, tais como em parques eólicos, na rede eléctrica ou outros sistemas físicos.

\bigskip

\textbf{Palavras chave:} deteção de anomalia, rede celular, deteção de falha, sistema fotovoltaico

\chapter*{Abstract}
%\addcontentsline{toc}{chapter}{Abstract}
The rise of global photovoltaic capacity has increased the need for reliable fault detection methods to minimize economic losses caused by downtime. These methods utilize advanced monitoring and diagnostic techniques to identify situations that can lead to decreased power generation or system failure, such as faulty/degraded panels or other components. By swiftly detecting and addressing faults, operators can minimize downtime, allocate resources efficiently, and optimize maintenance efforts. As the renewable energy sector expands, ongoing development and improvement of fault detection techniques are crucial for maintaining the efficiency and profitability of utility-scale photovoltaic power plants.

This work thoroughly evaluates fault detection tools' current status in photovoltaic (PV) systems. The goals are to understand how these tools work and identify their strengths and limitations. Based on the literature review, it is recognizable that machine learning techniques are the foundation of state-of-the-art fault detection and classification algorithms. However, it also shows the multidisciplinary nature of this field, with contributions from graph theory, signal processing, deep learning, and even emerging areas like quantum machine learning. Exploring these different approaches provides valuable insights into the advancements in fault detection for PV systems, opening doors for further developments and improvements in this research area.

Stemming from the acquired knowledge base, this work proposes a novel approach to tackle anomaly detection based on a diverging paradigm from what is seen in the literature. CellTAN (Cellular Time Activations Networks) is an asynchronous and distributed framework that leverages cell relationships to assess their state and identify anomalies. It boasts data privacy, component cooperation without data normalization, and independent operation. The capabilities introduced by this tool go beyond its application for PV systems, and it serves as a framework for anomaly detection in dynamic systems. It's tested against industrial-scale PV farm data (samples of two inverters and satellite) to validate anomaly detection capability and correctly assess specific situations.

The proposed approach yields promising results in identifying inverter anomalies based on their cell relationship and mapped underperformance scenarios. Its effectiveness is significant as it does not rely on specific system information nor breaks data privacy, enabling its application to diverse components from different stakeholders. This characteristic makes it versatile and applicable to any time-series system, increasing its potential impact on other fields. By demonstrating its efficacy in detecting anomalies and accommodating various system configurations, this tool offers a valuable solution for fault detection in diverse contexts, such as wind farms, the electrical grid, or other physical systems.


\bigskip

\textbf{Keywords:} anomaly detection, cell network, fault detection, photovoltaic system
