\chapter*{Abstract}
%\addcontentsline{toc}{chapter}{Abstract}

Message brokers provide asynchronous communication between data producers and
consumers in a distributed environment, being Kafka one of the several message
broker alternatives. To scale data consumption rate, Kafka has Consumer Groups,
which is an abstraction that allows to parallelize tasks between consumers in a
group. This abstraction presents new concerns that depend on the broker's
current load, which include: determining the number of consumers required;
determining the partition assignment between the consumers in a group in order
to guarantee that the consumption rate of each partition is not less than their
respective production rate. Additionally considering the load varies with time,
there is an increasing need to find autoscaling solutions to reduce operational
cost, while guaranteeing that all data is being consumed within acceptable time
after it has been produced.

As such, this problem is modeled as a new variation of the Bin Packing Problem
where the bins are consumers of a consumer group, and the weights are the
partitions and their respective write speeds. Due to the varying load applied to
Kafka brokers, the weights change in size over time. Another variation is the
fact that item assignments are not persistent, and therefore can be rebalanced
between consumers in different time instants. This adds another concern related
to the fact that while a partition is being rebalanced, data is not being
consumed from it (rebalance cost). We propose a new metric to account for this
cost (Rscore), and present four new heuristic algorithms based on the Rscore,
three of which prove to be a competitive alternative when compared to existing
heuristic algorithms with respect to the multi-objective optimization problem of
minimizing both the number of consumers and the rebalance cost.

To deliver a fully automated solution to the autoscaling problem, using the
aforementioned theoretical approach, we propose a system that is capable of
responding to a wide range of load applied to the brokers. When compared to
exisitng solutions, the proposed system is capable of reducing the operational
cost, and guarantees that the consumption rate keeps up with the production
rate, something that cannot be guaranteed by current solutions. 

\chapter*{Resumo}
%\addcontentsline{toc}{chapter}{Resumo}

\textit{Message Brokers} possibilitam a comunicação assíncrona entre produtores
e consumidores de informação num ambiente distribuído, sendo o Kafka uma das
várias ferramentas disponíveis para este tipo de comunicação. Com o intuito de
escalar a velocidade de consumo, o Kafka disponibiliza grupos de consumidores,
que é uma abstração que permite paralelizar tarefas entre os consumidores de um
grupo. Com esta abstração surgem as seguintes preocupações relacionadas com a
carga atual de um \textit{broker}: quantos consumidores são necessarios?; como é
que as partições vão ser distribuidas entre os consumidores de modo a garantir
que a velocidade de produção não excede a velocidade de consumo?. Considerando
que a carga dos \textit{Message Brokers} varia com o tempo, há uma crescente
necessidade de haver um sistema capaz de escalar um grupo de consumidores de
modo a garantir que não haja atrasos significativos no consumo dos dados
produzidos.

Modelou-se o problem como uma nova variação do \textit{Bin Packing Problem} (BPP), em
que os \textit{bins} são os consumidores, e os pesos são as partições e as suas
respetivas velocidades de escrita. Devido à variação da carga nos
\textit{brokers}, os valores dos pesos variam com o tempo. Outra variação deste
problema é que uma atribuição de uma partição a um consumidor não é persistente,
e pode ser redistribuida para outros consumidores. Isto apresenta uma nova
preocupação quando se procura uma solução a este problema associada ao facto de
que enquanto uma partição está a ser redistribuida, os seus dados não estão a
ser consumidos (custo de redistribuição). Propõe-se uma nova métrica (Rscore),
que procura quantificar o custo de redistribuição de uma nova configuração.
Baseados na nova metrica, apresenta-se também quatro novas heurísticas que
fornecem uma solução ao BPP tendo em conta o Rscore. Três destes quatro
algoritmos consideram-se soluções viáveis a este problema multi-objetivo de
minimizar tanto o numero de consumidores como o custo de redistribuição, quando
comparados a soluções existentes utilizando a frente de pareto.

Por forma disponibilizar uma solução autónoma capaz de escalar um grupo de
consumidores, apresenta-se também um sistema que tem a capacidade de resposta a
uma ampla variedade de carga dos \textit{brokers}. Quando comparado às soluções
existentes, o sistema consegue não só reduzir o custo operacional, como também
garantir que a velocidade de consumo acompanha a velocidade de produção, algo
que não é uma garantia nas soluções existentes.
