\chapter*{Resumo}
%\addcontentsline{toc}{chapter}{Resumo}

O aumento na quantia de centrais fotovoltaicas de escala industrial leva à necessidade de métodos de deteção de falhas fiáveis, de forma a minimizar perdas financeiras devidas a fecho de produção. Tais métodos utilizam técnicas avançadas de monitorização e diagnóstico para identificar anomalias, tais como painéis defeituosos ou componentes degradados, situações que possam levar à diminuição da produção de energia ou à falha do sistema. Ao detetar e resolver as falhas rapidamente, os operadores podem minimizar o tempo de inatividade do sistema, gerir os recursos de forma mais eficiente e otimizar os esforços de manutenção. A monitorização e análise contínua de um sistema permite a deteção precoce de anomalias, assegurando a produção ininterrupta de energia, com maior fiabilidade e resiliência, reduzindo o risco de falhas e interrupções em cascata. À medida que o sector das energias renováveis se expande, o desenvolvimento e melhoria contínuos nas técnicas de deteção de falhas são cruciais para manter a eficiência e a rentabilidade das centrais fotovoltaicas à escala industrial.

Neste trabalho encontra-se uma avaliação exaustiva do estado atual das ferramentas de deteção de falhas em sistemas fotovoltaicos (PV). O objetivo é compreender o funcionamento destas ferramentas e identificar quais os pontos fortes e limitações. Com base na revisão bibliográfica, reconhece-se que as técnicas de aprendizagem computacional são a base dos algoritmos mais avançados de deteção e classificação de falhas. Reconhecemos também a natureza multidisciplinar deste domínio, com contributos da teoria de grafos, processamento de sinal, aprendizagem profunda e até de áreas emergentes como a aprendizagem computacional quântica. Ao explorar as diferentes abordagens, relata-se os avanços na deteção de falhas e estimativa de estado para sistemas fotovoltaicos, traçando o caminho para novas formulações e melhorias nesta área de investigação.

Partindo da base de conhecimento adquirida, propõe-se uma nova abordagem para deteção de anomalias baseada num paradigma divergente da literatura. CellTAN (\textit{Cellular Time Activations Networks}) é uma estrutura assíncrona e distribuída que aproveita as relações entre células para avaliar o seu estado e identificar anomalias. Mantém a privacidade de dados, permite cooperação entre componentes sem normalização de dados e a sua operação independente.
Testa-se esta nova ferramenta com dados de parques fotovoltaicos de grande escala (amostra de dois inversores) para validar a sua capacidade de deteção de anomalias e situações específicas.

A abordagem proposta demonstra resultados promissores, na identificação de anomalias nos inversores com base nas relações celulares e cenários de perda de performance mapeados previamente. A eficácia do método é notável, uma vez que não depende de informações específicas do sistema nem quebra a privacidade dos dados, permitindo a sua aplicação a diversos tipos de componentes. Esta caraterística torna-o versátil e aplicável a qualquer sistema assente em séries temporais, aumentando a sua usabilidade noutros contextos e, consequentemente, potencial impacto. Ao demonstrar a sua eficácia na deteção de anomalias e ao acomodar várias configurações do sistema, esta ferramenta oferece uma solução de valor acrescido para a deteção de falhas em vários tipos de contexto, tais como em parques eólicos, na rede eléctrica ou outros sistemas físicos.

\chapter*{Abstract}
%\addcontentsline{toc}{chapter}{Abstract}
The rise of utility-scale photovoltaic power plants has increased the need for reliable fault detection methods to minimize economic losses caused by downtime. These methods utilize advanced monitoring and diagnostic techniques to identify anomalies, such as faulty panels or degraded components, which can lead to decreased power generation or system failure. By swiftly detecting and addressing faults, operators can minimize downtime, allocate resources efficiently, and optimize maintenance efforts. Continuous system monitoring and analysis enable early detection of anomalies, ensuring uninterrupted power generation, with enhanced reliability and resilience, reducing the risk of cascading failures and disruptions. As the renewable energy sector expands, ongoing development and improvement of fault detection techniques are crucial for maintaining the efficiency and profitability of utility-scale photovoltaic power plants.

This work presents a thorough evaluation of fault detection tools' current status in photovoltaic (PV) systems. The intention is to understand how these tools work and identify their strengths and limitations. Based on the literature review, we recognize that machine learning techniques are the foundation of state-of-the-art fault detection and classification algorithms. We also recognize the multidisciplinary nature of this field, with contributions from graph theory, signal processing, deep learning, and even emerging areas like quantum machine learning. By exploring these different approaches, we provide valuable insights into the advancements in fault detection and state estimation for PV systems, opening doors for further developments and improvements in this crucial research area.

Stemming from the acquired knowledge base, we propose a novel approach to tackle anomaly detection based on a diverging paradigm from what is seen in the literature. CellTAN (Cellular Time Activations Networks) is an asynchronous and distributed framework that leverages cell relationships to assess their state and identify anomalies. It boasts data privacy, component cooperation without data normalization, and independent operation.
We test it against industrial-scale PV farm data (samples of two inverters) to validate anomaly detection capability and correctly assess specific situations.

Our approach yields promising results in identifying inverter anomalies based on their cell relationship and mapped underperformance scenarios. The effectiveness of our method is noteworthy as it does not rely on specific system information nor breaks data privacy, enabling its application to diverse components from different stakeholders. This characteristic makes it versatile and applicable to any time-series system, enhancing its usability and potential impact. By demonstrating its efficacy in detecting anomalies and accommodating various system configurations, our approach offers a valuable solution for fault detection in diverse contexts, such as wind farms, the electrical grid, or other physical systems.

% By examining the most successful tools to date and offering a new solution, we contribute to improving the reliability and efficiency of PV systems. Furthermore, our generalistic approach contributes to anomaly detection for other fields.

\smallskip

\textbf{Keywords:} fault detection, photovoltaics, anomaly detection, cell network