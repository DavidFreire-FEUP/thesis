\chapter*{Resumo}
%\addcontentsline{toc}{chapter}{Resumo}

A proliferação de centrais fotovoltaicas de dimensão industrial leva à necessidade de métodos para detetar e classificar falhas nos seus componentes, sendo que estas que podem ter impactos económicos significativos. Neste trabalho, o estado da arte das ferramentas de deteção de falhas e estimação do estado aplicadas ao campo dos sistemas PV será explorado, com foco na compreensão do seu funcionamento, identificando-se pontos fortes e possíveis limitações. Conclui-se que os métodos estatísticos não são comummente utilizados nas ferramentas modernas. Ainda assim, já foi testada a implementação de diversos domínios para solucionar este tipo de problema, desde teoria dos grafos a processamento de sinal, aprendizagem profunda e aprendizagem quântica. Serão propostas melhorias às abordagens existentes ou desenvolvida uma nova abordagem para abordar esta problemática. Com a inspeção das ferramentas mais bem sucedidas até à data e pela potencial oferta de uma nova abordagem, o objetivo deste trabalho é fornecer aos operadores de instalações fotovoltaicas um aumento na fiabilidade e eficiência dos seus sistemas.


\chapter*{Abstract}
%\addcontentsline{toc}{chapter}{Abstract}

The increase in utility-scale photovoltaic power plants has led to the need for effective methods for detecting and classifying component faults, which can have significant economic impacts. This work assesses the current state of fault detection and state estimation tools in the field of PV systems, focusing on understanding how these tools work and identifying their strengths and limitations. It is concluded that statistical methods are not commonly used on modern tools, while machine learning makes up the majority of state-of-the-art fault detection and classification algorithms. Still, many fields have been tested for this problem, from graph theory to signal processing, deep learning, and quantum machine learning. Consequently, this work proposes improvements to existing approaches or a novel technique developed to address this issue. By examining the most successful tools to date and offering new solutions, the intention is to help PV plant operators improve the reliability and efficiency of their systems. Also, it's expected that the developed methodology can become a generalistic data cohesion algorithm, positively impacting other data-driven problems.